	\section{UMLet}
	Til beskrivelse af systemarkitektur og HW- og SW-design benyttes tegneværktøjet UMLet. Værktøjet indeholder en række figurer inddelt i kategorier, som gør det forholdsvist let at lave f.eks. applikationsmodeller og andre diagrammer og modeller, som kan være relevante for projektet, mens det samtidig tilbyder en stor fleksibilitet i redigering af de diverse figurer, som det stiller til rådighed, samt giver brugeren muligheden for at danne sine egne figurer vha. Java-kode. Dermed egner UMLet sig ikke kun til at danne UML-diagrammer, som det primært er beregnet til, men det udmærker sig også til SysML-diagrammer. Designet af diagrammer sker ved hjælp af "drag-and-drop"-metoden, hvor figurer placeres i et overordnet terræn, og hvor disse derefter kan manipuleres med ved f.eks. at trække i dem for at ændre på størrelsen og formen eller ved at ændre deres attributter og form i et tekstfelt.
