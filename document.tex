
\documentclass[a4paper,article,11pt,oneside]{memoir}
\usepackage[UTF8]{inputenc} 
\usepackage{hyperref}
\setcounter{secnumdepth}{4}

%opening
\title{Samarbejdsaftale}
\author{Tonni Nybo Follmann \and Stefan Nielsen \and Mikkel Busk Espersen \and Ahmad Sabah \and Halfdan Vanderbruggen Bjerre \and Jacob Munkholm Hansen}

\begin{document}

\maketitle

\begin{abstract}
Samarbejdsaftale for Semester projekt 3 Gruppe 13.
\end{abstract}


\frontmatter
\tableofcontents


\mainmatter
\chapter{Indledning}

\chapter{Ambitionsniveau}

\section{Karakterer}
Vi går efter 12-tallet.
\section{Arbejdstid}
Der forventes en arbejdsbyrde på ca. 10 timer ugentligt per person + lektier.

Der vil som regel blive givet lektier for til hvert møde og disse forventes selvfølgelig læst til aftalt tidspunkt.
\section{Mødetider og -indkaldelser}
I første sprint afholdes der 2 ugentlige møder - mandag og onsdag. Om mandagen vil gruppen mødes kl. 17-21 primært for at arbejde samlet og på tværs af studieretninger.
Hver anden onsdag samles gruppen for at afholde et Scrum review-møde og derefter planlægge næste sprint.
Desuden afholdes tirsdag og fredag ca. kl. 13.55-14.10 Scrum stå-op-møder hvor arbejdsopgaver diskuteres.

De fleste møder er faste, men skulle et møde opstå uventet skal dette være hele gruppen oplyst senest kl. 18 dagen før mødet afholdes.

\section{Ansvar}
Det forventes at man selv tager ansvar for sine arbejdsopgaver og melder ud hvis der er behov for hjælp.
\chapter{Rollefordeling}
\paragraph{Product Owner}
Alle medlemmerne i gruppen skal påtage sig rollen som product owner.
\paragraph*{Overordnet Scrummaster} Mikkel Busk
\paragraph*{Suppleant Scrummaster} Jacob Hansen
\paragraph*{Mødeleder}
Mikkel Busk
\paragraph*{Fast Referent}
Halfdan 
\paragraph*{Suppleant Referent}
Stefan
\chapter{Disciplin}
\section{Fremmøde}
Det forventes at gruppemedlemmerne møder ind til den aftalte tid ved alle møder.
Ved forsinkelse gives der besked inden mødestart.  Ved mere end  5 minuttes forsinkelse uden rettidig meddelelse herom gives der 1 advarsel. Ved 3 advarsler afholdes der afstemning med henblik på ekskludering af det enkelte gruppemedlem. Afstemningen afgøres ved simpelt flertal, ved lige stemmefordeling afgøres det af vejlederen.

Ved manglende deltagelse ved et møde forventes deltagelse ved næste møde samt at referater fra sidste møde er læst igennem og forstået. Sker dette ikke uden anden aftale med gruppen, udløser det en advarsel på lige fod med udeblivelse uden rettidig meddelelse.

\section{Arbejdsopgaver}
Det forventes at de arbejdsopgaver man påtager sig er færdige til den aftalte tid. Er dette ikke muligt skal dette meddeles gruppen senest 24 timer før deadline for arbejdsopgaven. Sker dette ikke gælder der samme regler som ved forsinket mødetider.
\chapter{Proces og udvikling}
Gruppen følger ASE-modellen (Aarhus School of Engineering) som skabelon for den overordnede udviklingsproces. Til at strukturere udviklingsprocessen i det daglige benyttes SCRUM. På denne måde gøres store dele af ASE-modellen iterativ og der benyttes begreber fra såvel ASE-modellen og SCRUM i både tale og på skrift.
Der dokumenteres løbende, dvs. hver gang et sprint er overstået indføres dokumentation for det udførte arbejde i et fælles arkiv (GitHub). Denne dokumentation er som regel individuel og altid på eget ansvar. Logbog føres for hver gang et stykke arbejde er blevet udført og dette ansvar pålægges det enkelte gruppemedlem – dvs. der føres individuelle logbøger og alle er på eget ansvar at få skrevet. Det forventes at alle gruppemedlemmer fører logbog som det første efter endt arbejdsdag.
Dokumentation samt logbog føres efter aftalte skabeloner.
\chapter{Udarbejdelse af dokumenter}

\section{Møde Indkaldelser og referater}
Mødeindkaldelser og referater skrives i MS Word.

\section{Anden Dokumentation}
Alt dokumentation skrives i LaTeX da MS Word er ekstrem dårlig til at håndtere store dokumenter. Her er LaTeX tilgengæld rigtig godt at arbejde i, og da LaTeX i sig selv er simple tekst filer giver det ligeledes bedre understøttelse af versionskontrolsystemet GIT.
\subsection{Design af Dokumentation}
Der anvendes klassen Memoir til at lave layout i dokumenterne, for yderligere infomation om Memoir se \url{ftp://tug.ctan.org/pub/tex-archive/macros/latex/contrib/memoir/memman.pdf}
\subsection{Software til at skrive LaTex i}
Til at skrive LaTeX er det valgfrit hvilket værktøj der benyttes så længe det understøtter UTF8 Encoding, da dette giver den bedste understøttelse af nordiske tegn.
Et par eksempler på gode LaTeX editore er:

\begin{itemize}
	\item TeXstudio - \url{http://www.texstudio.org/}
	\item TeXmaker - \url{http://www.xm1math.net/texmaker/}
	\item TeXworks - \url{https://www.tug.org/texworks/}
\end{itemize}
Der er mange andre gode editors der ude, så vælg den der føles mest rigtig for dig.

\subsection{Guide til LaTeX, samt vejledning i skrivning af projekter i LaTeX}
Er man i tvivl om noget ved brugen af LaTeX er der på AU udgivet en god dansk bog der indeholder en del hints til brugen af memoir klassen samt dokumentopbygning.

Det forventes at denne bog skimtes igennem, og benyttes som reference sammen med memoir klassens dokumentation når der oprettes dokumenter til projektet.
Specielt side 331 med dødsynder skal læses og forstås.

Links til bogen og yderligere hjælp findes på \url{http://math.au.dk/videnudveksling/latex/}
\chapter{Anvendte Værktøjer}
\section{RedMine}
Vi har på opfordring fra vores vejleder valgt at anvende Redmine. 
Redmine er en open source webbaseret applikation, som sammen med en masse tredjepartsprogrammer kan anvendes til projektstyring. Her kan oprettes issues, som indeholder opgaver, del-opgaver eller underpunkter som skal løses i forbindelse med udviklingen af projektet, disse opgaver tildeles så en forventet tid hvorefter en eller flere personer kan gå i gang med udviklingen eller undersøgelserne. Udviklerne holder på Redmine hjemmesiden styr på deres tidsforbrug på de forskellige opgaver med funktionen "log time", og redmine bruger så alt dette til at generere et burndown chart, som skal hjælpe os med at nå målet i tide. Der er i Redmine også mulighed for at anvende Wiki. Vi har valt i projektgruppen at anvende Wiki til at udfører personlig logbog over udført arbejde og tid, samt til at noterer alle ting vi finder ud af i vores indledende undersøgelser.
\section{Slack}
Vi anvender i projektgruppen "Slack" som primært kommunikations medie. Det er i Slack muligt at integrerer en hel del andre programmer som bl.a. Google kalender og GitHub så der vil komme notifikationer i Slack appen når der sker noget nyt. Vi har valgt at udnytte disse muligheder, for at samle al kommunikation et sted, for på den måde at mindste muligheden for misforståelser.
\section{Google kalender}
Der er oprettet en Google kalender som er projektets kalender, denne er så blevet integreret med Slack, så nye begivenheder der oprettes i kalenderen afføder en notifikation på Slack. Det er dessuden muligt for projekt gruppens medlemmer at tilføje projekt gruppens kalender på deres mobiltelefon, og så vil man på den måde altid kunne se hvis der er et møde eller andet planlagt.
\section{GitHub}
vi har besluttet at anvende Github til at holde styr på vores dokumenter, frem for det indbyggede repository system i RedMine systemet, da vi har mulighed for at integrerer GitHub med vores Slack channel, således der kommer opdateringer til hele holdet når der ændres eller tilføjes dokumenter.
\input{umlet}
\end{document}
