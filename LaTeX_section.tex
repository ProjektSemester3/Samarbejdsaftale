\chapter{Udarbejdelse af dokumenter}

\section{Møde Indkaldelser og referater}
Mødeindkaldelser og referater skrives i MS Word.

\section{Anden Dokumentation}
Alt dokumentation skrives i LaTeX da MS Word er ekstrem dårlig til at håndtere store dokumenter. Her er LaTeX tilgengæld rigtig godt at arbejde i, og da LaTeX i sig selv er simple tekst filer giver det ligeledes bedre understøttelse af versionskontrolsystemet GIT.
\subsection{Design af Dokumentation}
Der anvendes klassen Memoir til at lave layout i dokumenterne, for yderligere infomation om Memoir se \url{ftp://tug.ctan.org/pub/tex-archive/macros/latex/contrib/memoir/memman.pdf}
\subsection{Software til at skrive LaTex i}
Til at skrive LaTeX er det valgfrit hvilket værktøj der benyttes så længe det understøtter UTF8 Encoding, da dette giver den bedste understøttelse af nordiske tegn.
Et par eksempler på gode LaTeX editore er:

\begin{itemize}
	\item TeXstudio - \url{http://www.texstudio.org/}
	\item TeXmaker - \url{http://www.xm1math.net/texmaker/}
	\item TeXworks - \url{https://www.tug.org/texworks/}
\end{itemize}
Der er mange andre gode editors der ude, så vælg den der føles mest rigtig for dig.

\subsection{Guide til LaTeX, samt vejledning i skrivning af projekter i LaTeX}
Er man i tvivl om noget ved brugen af LaTeX er der på AU udgivet en god dansk bog der indeholder en del hints til brugen af memoir klassen samt dokumentopbygning.

Det forventes at denne bog skimtes igennem, og benyttes som reference sammen med memoir klassens dokumentation når der oprettes dokumenter til projektet.
Specielt side 331 med dødsynder skal læses og forstås.

Links til bogen og yderligere hjælp findes på \url{http://math.au.dk/videnudveksling/latex/}