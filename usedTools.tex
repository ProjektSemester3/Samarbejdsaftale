\chapter{Anvendte Værktøjer}
\section{RedMine}
Vi har på opfordring fra vores vejleder valgt at anvende Redmine. 
Redmine er en open source, webbaseret applikation som kan anvendes til projektstyring. Her kan oprettes issues, som indeholder opgaver, del-opgaver eller underpunkter som skal løses i forbindelse med udviklingen af projektet, disse opgaver tildeles så en forventet tid hvorefter en eller flere personer kan gå i gang med udviklingen eller undersøgelserne. Udviklerne holder på Redmine hjemmesiden styr på deres tidsforbrug på de forskellige opgaver med funktionen "log time", og redmine bruger så alt dette til at generere et burndown chart, som skal hjælpe os med at nå målet i tide. Der er i Redmine også mulighed for at anvende Wiki. Vi har valt i projektgruppen at anvende Wiki til at udfører personlig logbog over udført arbejde og tid, samt til at noterer alle ting vi finder ud af i vores indledende undersøgelser.
\section{Slack}
Vi anvender i projektgruppen "Slack" som primært kommunikations medie. Det er i Slack muligt at integrerer en hel del andre programmer som bl.a. Google kalender og GitHub så der vil komme notifikationer i Slack appen når der sker noget nyt. Vi har valgt at udnytte disse muligheder, for at samle al kommunikation et sted, for på den måde at mindste muligheden for misforståelser.
\section{Google kalender}
Der er oprettet en Google kalender som er projektets kalender, denne er så blevet integreret med Slack, så nye begivenheder der oprettes i kalenderen afføder en notifikation på Slack. Det er dessuden muligt for projekt gruppens medlemmer at tilføje projekt gruppens kalender på deres mobiltelefon, og så vil man på den måde altid kunne se hvis der er et møde eller andet planlagt.
\input{umlet}